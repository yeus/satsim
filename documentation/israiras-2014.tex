%%%%%%%%%%%%%%%%%%%%%%%%%%%%%%%%%%%%%%%%%%%%%%%%%%%%%%%%
%
%  LaTeX Document for i-SAIRAS 2014 in St-Hubert, Canada
%
%%%%%%%%%%%%%%%%%%%%%%%%%%%%%%%%%%%%%%%%%%%%%%%%%%%%%%%%
\documentclass[letter,twocolumn]{article}
\usepackage{isairas}
\usepackage[utf8x]{inputenc}
\usepackage{indentfirst}
\usepackage{txfonts}
\usepackage[dvipdfm]{graphicx,color}
%\usepackage{python}

\begin{document}
\title{Autonomous Adaption of AOCS in Modular, Reconfigurable Satellites}
\subtitle{}
\author{T. Meschede*, J. Rießelmann**}
\affiliation{{*}Department of Astronautics, TU Berlin, Germany\\
e-mail: Thomas.Meschede@tu-berlin.de\vspace{2mm}\\
{**}Department of Astronautics, TU Berlin, Germany\\
e-mail: Jens.Riesselmann@tu-berlin.de}

\maketitle
\thispagestyle{empty}

%%%%%%%%%%%%%%%%%%%%%%%%%%%%%%%%%%%%%%%%%%%%%%%%%%%%%%%%%%%%%%%%%%%%%%%%%%%%%%%%
\begin{abstract}
Use the word ``Abstract'' as the title, in 12-point Times, boldface type, centered relative to the column, initially capitalized. The abstract is to be in 10-point, single-spaced type, and up to 150 words in length. The abstract is an essential part of the paper. Use short, direct, and complete sentences. It should be as brief as possible and concise. It should be complete, self-explanatory, and not require reference to the paper itself. The abstract should be informative giving the scope and emphasize the main conclusions, results, or significance of the work described. Do not use the first person; do not include mathematical expressions; do not refer to the reference, and try to avoid acronyms.
\end{abstract}

- list()

%%%%%%%%%%%%%%%%%%%%%%%%%%%%%%%%%%%%%%%%%%%%%%%%%%%%%%%%%%%%%%%%%%%%%%%%%%%%%%%%
\section{Introduction}
All manuscripts must be in English. These guidelines include complete descriptions of the fonts, spacing, and related information for producing your proceedings manuscripts. Please follow them and if you have any question, contact to the i-SAIRAS 2014 secretariat by e-mail, (isairas2014@asc-csa.gc.ca).

%%%%%%%%%%%%%%%%%%%%%%%%%%%%%%%%%%%%%%%%%%%%%%%%%%%%%%%%%%%%%%%%%%%%%%%%%%%%%%%%
\section{Formatting Instructions}
All printed material, including text, illustrations, and charts, must be kept within a print area of 6-1/2 inches (approximately 17 cm) wide by 9 inches (approximately 23 cm) high. Do not write or print anything outside the print area. All text must be in a two-column format. Columns are to be 3-3/4 inches (approximately 8.1 cm) wide, with a 3/8 inch (approximately 0.81 cm) space between them. Text must be fully justified.

\subsection{Length and Size}
Total paper length should not exceed 8 pages. The size should be less than 3MB.

\subsection{Main title}
The main title (on the first page) should begin from the top edge of the page,, centered, and in Times 14-point, boldface type. Capitalize the first letter of nouns, pronouns, verbs, adjectives, and adverbs; do not capitalize articles, coordinate conjunctions, or prepositions (unless the title begins with such a word). Leave two 12-point blank lines after the title.

\subsection{Author name(s) and affiliation(s)}
Author names and affiliations are to be centered beneath the title and printed in Times 12-point, non-boldface type. Multiple authors may be shown in a two- or three-column format, with their affiliations italicized and centered below their respective names. Include e-mail addresses if possible. Author information should be followed by one blank line.

\subsection{Type-style and fonts}
Wherever Times is specified, Times Roman or Times New Roman may be used. If neither is available on your word processor, please use the font closest in appearance to Times. Avoid using bit-mapped fonts if possible. True-Type 1 fonts are preferred.

%%%%%%%%%%%%%%%%%%%%%%%%%%%%%%%%%%%%%%%%%%%%%%%%%%%%%%%%%%%%%%%%%%%%%%%%%%%%%%%%
\section{Main Text}
Type your main text in 10-point Times, single-spaced. Do not use double-spacing. All paragraphs should be indented 1/4 inch (approximately 0.6cm). Be sure your text is fully justified---that is, flush left and flush right. Please do not place any additional blank lines between paragraphs.

\subsection{Tables}

All tables must be centered in the column and numbered consecutively (in Arabic numbers). Table headings should be placed above the table, in 10-point boldface Times. Place tables as close as possible to where they are mentioned in the main text.

\subsection{Figures}
Since the proceedings will be published electronically, high-resolution colour images are preferred. All illustrations must be numbered consecutively (i.e., not section-wise), using Arabic numbers. Figure captions should be 10-point boldface Times. Initially capitalize only the first word of each figure caption. Figures and tables must be numbered separately. For example: ``Figure 1. Database contexts'', ``Table 1. Input data''. Figure captions  are to be centered below the figures. Figures should appear in the body of the text as close as possible to the in-text call-out reference (and not at the end of the paper).

\subsection{Mathematical formula}
Mathematical formulas should be roughly centered and have to be numbered as formula (1).

\subsection{Headings}
The first-order headings should be Times 12-point boldface, initially capitalized, flush left, with one blank line before, and one blank line after. The second-order headings should be Times 11-point boldface, initially capitalized, flush left, with one blank line before, and one after. Third-order headings, as in this paragraph, are discouraged. However, if you must use them, use 10-point Times, boldface, initially capitalized, flush left, preceded by one blank line.

\subsection{Footnotes}
Use footnotes sparingly (or not at all) and place them at the bottom of the column on the page on which they are referenced. Use Times 8-point type, single-spaced. To help your readers, avoid using footnotes altogether and include necessary peripheral observations in the text (within parentheses, if you prefer, as in this sentence).

\subsection{Acknowledgment}
The preferred spelling of the word ``acknowledgment'' in America is without an ''e'' after the ``g''. Try to avoid the stilted expression, ``One of use (R.B.G.) thanks $\cdots$'' Instead, try ``R.B.G. thanks $\cdots$'' Sponsor acknowledgements may also be included here.

\subsection{References}
List and number all bibliographical references in 10-point Times, single-spaced, at the end of your paper. When referenced in the text, enclose the citation number in square brackets, for example \cite{smith}. Where appropriate, include the name(s) of editors of referenced books.

%%%%%%%%%%%%%%%%%%%%%%%%%%%%%%%%%%%%%%%%%%%%%%%%%%%%%%%%%%%%%%%%%%%%%%%%%%%%%%%%
\section{Conclusions}
After proofreading your paper, it must be submitted electronically using PDF formats as per instructions to authors appearing
on the i-SAIRAS 2014 web site :\\\indent
(http://www.asc-csa.gc.ca/eng/events/2014/i-sairas.asp)\\\noindent
\\\indent
Please use the paper number for file name, which you received when you submitted the abstract, for example, 123.pdf.\\\indent
Do not send hard copies or use other file formats they will not be accepted. Proper usage of the English language is expected of all submissions (i.e., Camera-ready papers). Make sure that the PDF file looks fine on the screen as well as in print.

%%%%%%%%%%%%%%%%%%%%%%%%%%%%%%%%%%%%%%%%%%%%%%%%%%%%%%%%%%%%%%%%%%%%%%%%%%%%%%%%
\begin{thebibliography}{99}
\bibitem{smith}
A.B. Smith, C.D. Jones, and E.F. Roberts, ``Article Title'', Journal, Publisher, Location, Date, pp. 1-10.

\vspace{-1.5mm}
\bibitem{jones}
C.D. Jones, A.B. Smith, and E.F. Roberts, Book Title, Publisher, Location, Date.

\end{thebibliography}

\end{document}